
\documentclass[margin,line]{res}


\oddsidemargin -.5in
\evensidemargin -.5in
\textwidth=6.0in
\itemsep=0in
\parsep=0in
% if using pdflatex:
%\setlength{\pdfpagewidth}{\paperwidth}
%\setlength{\pdfpageheight}{\paperheight}

\newenvironment{list1}{
  \begin{list}{\ding{113}}{%
      \setlength{\itemsep}{0in}
      \setlength{\parsep}{0in} \setlength{\parskip}{0in}
      \setlength{\topsep}{0in} \setlength{\partopsep}{0in}
      \setlength{\leftmargin}{0.17in}}}{\end{list}}
\newenvironment{list2}{
  \begin{list}{$\bullet$}{%
      \setlength{\itemsep}{0in}
      \setlength{\parsep}{0in} \setlength{\parskip}{0in}
      \setlength{\topsep}{0in} \setlength{\partopsep}{0in}
      \setlength{\leftmargin}{0.2in}}}{\end{list}}


\begin{document}

\begin{flushleft}
		\huge \textbf{Kristin M. Eccles, Ph.D.}
		\hfill {\LARGE \textbf{Resume}}\\
\end{flushleft}

\begin{resume}

\section{\sc Contact Information}
\vspace{.05in}
\begin{tabular}{@{}p{3.2in}p{2.5in}}
{\it Location:} Durham, North Carolina, USA  &{\it Website:} https://kristineccles.wordpress.com  \\
{\it Voice:}  (919) 998-9954 & {\it GitHub:} https://github.com/kristineccles \\
{\it E-mail:}  kristin.eccles@gmail.com \\

\end{tabular}

\section{\sc Skills and Comptencies}
\begin{itemize}
\item 8 years of  research experiences conceptualizing and designing studies, identifying testable hypotheses, collecting data, performing statistical analysis, and communicating results through written peer-reviewed publications and oral presentations at conferences
\item Developing, managing, and statistically analyzing large multi-dimensional health datasets
\item Quantifying data and analysis precision and accursing using quality assurance/ quality control and validation methods
\item Good laborary practice and good clinical practice certified
\item Experience generating and using biomarkers of chemical exposure and biological effect using thermal decomposition, ICP-MS, LC-MS, and high throughput screening assays
\item Languages: English - Native Language, French - Good, R - Advanced, Python - Intermediate, LaTeX- Intermediate
\end{itemize}

\vspace*{.1in}
\section{\sc Employment}

{\bf  Postdoctoral Research Fellow} \hfill {\bf Nov 2020 - Present}\\
National Institute of Environmental Health Science, Division of the National Toxicology Program,\\
Durham, North Carolina, USA
\begin{itemize}
\item Conceptualized and carried out computational research to determine the population-based risk of exposure to chemical mixtures by integrated environmental chemical exposure data with chemical hazard data to quantify how local exposure to chemical mixtures can perturb common biological pathways leading to adverse health outcomes.
\end{itemize}

{\bf Postdoctoral Research Fellowship} \hfill {\bf Aug 2019 - Oct 2020}\\
University of Toronto, Mississauga, Canada
\begin{itemize}
\item Designed and carried out research on historical concentrations of environmental contaminants by generating mercury data in natural archives (e.g., tree rings, sediment cores) and using statistical approaches such as time series analysis to identify patterns of environmental contaminants over time.
\end{itemize}

{\bf Geomatics Researcher} \hfill {\bf June 2017 - March 2019}\\
Environment and Climate Change Canada, Ottawa, Canada
\begin{itemize}
\item Synthesizes all geospatial contaminants and biomarker data in wildlife from the Canadian Oil Sands.
\item Conducted geospatial and statistical analysis to quantify over arching spatial patterns of contaminants in the Oils Sands.
\item Use evidence-based decision-making from research results to collaboratively design future biomonitoring programs.
\end{itemize}

{\bf Course Instructor} \hfill {\bf Sept 2016 - June 2020}\\
Geographic Information Systems (University of Toronto, Canada), Class Size: 50\\
Introduction to Quantitative Methods (Carleton University, Canada), Class size: 150\\
Introduction to Geomatics (University of Ottawa, Canada), Class size: 40\\
\begin{itemize}
\item Provided theoretical and practical training for students on data collection, geospatial and statistical analysis methods.
\item Designed the courses, prepared the lecture slides, delivered the lectures (3 hours of lecturing per week), designed practical labs and oversaw implementation, designed course evaluations and exams to determine if learning objective had been met.
\end{itemize}

\vspace*{.1in}
\section{\sc Education}

\textbf{Ph.D., Biology with Specialization in Chemical and Environmental Toxicology} \hfill {\bf 2019}\\
University of Ottawa, Ottawa, Canada

\textbf{M.Sc., Geography}  \hfill {\bf 2014}\\
University of Calgary, Calgary, Canada

\textbf{Honours B.A., Major: Health Studies, Minors: Geography and Earth Science }\hfill {\bf 2012}\\
McMaster University, Hamilton, Canada

\vspace*{.1in}

\section{\sc Peer-reviewed Publications}
\textbf{Summary: 18 journal publications (9 as first author), 248 citations; h index = 9 and i10 index = 9}

Cui, Y., \textbf{Eccles K.M.}, Kwok, R.K., Joubert, B., Messier, K.P., Balshaw, D. (2022). Integrating Multiscale Geospatial Environmental Data into Large Population Health Studies: Challenges and Opportunities. \textit{Toxics}. 10(403). https://doi.org/10.3390/toxics10070403

Thomas, P. J., Eickmeyer, D. C., \textbf{Eccles, K.M.}, Kimpe, L. E., Felzel, E., Brouwer, A., Blais, J. M. (2022). Paleotoxicity of petrogenic and pyrogenic hydrocarbon mixtures in sediment cores from the Athabasca oil sands region, Alberta (Canada). \textit{Environmental Pollution}, 118271. \\
https://doi.org/10.1016/j.envpol.2021.118271

\textbf{Eccles, K.M.}, Thomas, P. J., Chan, H. M. (2021). Spatial patterns of the exposure-response relationship between mercury and cortisol in the fur of river otter (\textit{Lontra canadensis}). \textit{Chemosphere}, 263, 127992. https://doi.org/10.1016/j.chemosphere.2020.127992

Thomas, P. J., Newell, E. E., \textbf{Eccles, K.M.}, Holloway, A. C., Idowu, I., Xia, Z., Quenneville, C. (2021). Co-exposures to trace elements and polycyclic aromatic compounds (PACs) impacts North American river otter (\textit{Lontra canadensis}) baculum. \textit{Chemosphere}, 265, 128920.\\https://doi.org/10.1016/j.chemosphere.2020.128920

\textbf{Eccles, K.M.}, Pauli, B.D., Chan, H.M. (2020). Geospatial analysis of complex metal exposures to biota in the Athabasca Oil Sands. \textit{PLoS one}, 15(9), e0239086. \\https://doi.org/10.1371/journal.pone.0239086

Galen, G., \textbf{Eccles, K.M}., MacMillian, M., Thomas, P. J., Chan, H.M., Poulain, A.J. (2020). The gut microbial community structure of the North American river otter (\textit{Lontra canadensis}) in the Alberta Oil Sands Region in Canada: relationship with local environmental variables and metal body burden.\textit{Environmental toxicology and chemistry}.39(12), 2516-2526. https://doi.org/10.1002/etc.4876

Etowa, J., Johnston, A., Jama, Z., \textbf{Eccles, K.M.}, Ashton, A. (2020). Mixed-method evaluation of a community-based postpartum support program: a study protocol. \textit{BMJ open}, 10(10), e036749. https://doi.org/10.1136/bmjopen-2019-036749

\textbf{Eccles, K.M.}, Majeed, H., Lehnherr, I., Porter, T. (2020). A continental and marine-influenced tree-ring mercury record in the Old Crow Flats, Yukon, Canada. \textit{ACS Earth and Space Chemistry}, 4(8), 1281-1290. https://doi.org/10.1021/acsearthspacechem.0c00081.s001

Cheney, C.L., \textbf{Eccles, K.M}., Kimpe, L.E., Blais, J.M. (2020). Determining the effects of past gold mining using a sediment palaeotoxicity model. \textit{Science of The Total Environment}, 718, 137308. https://doi.org/10.1016/j.scitotenv.2020.137308

\textbf{Eccles, K.M.}, Thomas, P. J., Chan, H. M. (2020). Relationships between mercury concentrations in fur and stomach contents of river otter (\textit{Lontra canadensis}) and mink (\textit{Neovison vison}) in northern Alberta Canada and their applications as proxies for environmental factors determining mercury bioavailability. \textit{Environmental Research}, 181, 108961. https://doi.org/10.1016/j.envres.2019.108961

\textbf{Eccles, K. M}., Pauli, B. D., Chan, H. M. (2019). The use of Geographic Information Systems (GIS) for spatial ecological risk assessments: An example from the Athabasca oil sands area in Canada. \textit{Environmental toxicology and chemistry}, 38(12): 2797–2810. https://doi.org/10.1002/etc.4577

\textbf{Eccles, K. M.}, Littlewood, E. S., Thomas, P. J., Chan, H. M. (2019). Distribution of organic and inorganic mercury across the pelts of Canadian river otter (\textit{Lontra canadensis}). \textit{Scientific reports}, 9(1), 3237. https://doi.org/10.1038/s41598-019-39893-w

\textbf{Eccles, K. M.}, Thomas, P. J., Chan, H. M. (2017). Predictive meta-regressions relating mercurytissue concentrations of freshwater piscivorous mammals. \textit{Environmental Toxicology and Chemistry}, 36(6), 2377–2384. http://doi.org/10.1002/etc.3775

Thomas, P. J., \textbf{Eccles, K. M.}, Mundy, L. J. (2017). Spatial modelling of non-target exposure to anticoagulant rodenticides can inform mitigation options in two boreal predators inhabiting areas with intensive oil and gas development. \textit{Biological Conservation}, 212, 111-119. \\https://doi.org/10.1002/etc.3775

Hu, X. F., \textbf{Eccles, K. M.}, Chan, H. M. (2017). High selenium exposure lowers the odds ratios for hypertension, stroke, and myocardial infarction associated with mercury exposure among Inuit in Canada. \textit{Environment International}, 102, 200-206. https://doi.org/10.1016/j.envint.2017.03.002

\textbf{Eccles, K. M.}, Checkley, S., Sjogren, D., Barkema, H. W., Bertazzon, S. (2017). Lessons learned from the 2013 Calgary flood: Assessing risk of drinking water well contamination. \textit{Applied Geography}, 80, 78-85. https://doi.org/10.1016/j.apgeog.2017.02.005

\textbf{Eccles, K.M.}, Bertazzon, S. (2015). Applications of geographic information systems in public health: A geospatial approach to analyzing MMR immunization uptake in Alberta. \textit{Canadian Journal of Public Health}, 106(6). https://doi.org/10.17269/cjph.106.4981

Bertazzon, S., Johnson, M., \textbf{Eccles, K.}, Kaplan, G. G. (2015). Accounting for spatial effects in land use regression for urban air pollution modelling. \textit{Spatial and Spatio-temporal Epidemiology}. 14-15, 9–21. https://doi.org/10.1016/j.sste.2015.06.002

\section{\sc Manuscripts in Progress}

\textbf{Eccles K.M.}, Karmaus A.L., Parham, F., Rider, C.V., Kleinstreuer, N.C., Wambaugh, J.F., Messier, K.P. A geospatial modeling approach to quantifying risk of exposure to environmental chemical mixtures via a common molecular target. (In Peer Review)

Lowe, M.E., Akhtari, F., Potter, P.A., Fargo, D.C., Schmitt, C.P., Schurman, S.H., \textbf{Eccles, K.M.}, Motsinger-Reif, A., Hall, J.E., Messier, K.P.The skin is no barrier to mixtures: Air pollutant mixtures and reported psoriasis or eczema in the Personalized Environment and Genes Study (PEGS). (In Peer Review)

Boutet, V., Dominique, M., \textbf{Eccles, K.M.,}, Branigan, M., Dyck, M., van Coeverden de Groot, P., Lougheed, S.C., Rutter A., Langlois V.S. A spatial contaminant assessment for polar bear (Ursus maritimus) liver, fat, and muscle from Northern Canada. (In Revisions)

Schock, D.M., Akhter, F., Mundy, L., \textbf{Eccles K.M.}, Soos, C., Papp, Z., Pauli., B. Spatial and temporal variation of metal concentrations in wetlands and wood frogs from the Athabasca oil sands region in northern Alberta, Canada. (In Peer Review)\\

Tommasi, F., Pagano, G., Oral, R., Thomas, P.J., \textbf{Eccles K.M.}, Tez, S., Toscanesi, M., Giarra,, A., Siciliano, A., Dipierro, N., Gjata, I,, Guida, M., Libralato, G., Lyons,D.M., Burić, P., Ines Kovačić, I.,  Trifuoggi, M. Topsoil pollution and multi-endpoint toxicity in the petrochemical area of Augusta-Priolo (eastern Sicily, Italy). (In Peer Review)\\

\end{resume}
\end{document}




