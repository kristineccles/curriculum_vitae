
\documentclass[margin,line]{res}


\oddsidemargin -.5in
\evensidemargin -.5in
\textwidth=6.0in
\itemsep=0in
\parsep=0in
% if using pdflatex:
%\setlength{\pdfpagewidth}{\paperwidth}
%\setlength{\pdfpageheight}{\paperheight}

\newenvironment{list1}{
  \begin{list}{\ding{113}}{%
      \setlength{\itemsep}{0in}
      \setlength{\parsep}{0in} \setlength{\parskip}{0in}
      \setlength{\topsep}{0in} \setlength{\partopsep}{0in}
      \setlength{\leftmargin}{0.17in}}}{\end{list}}
\newenvironment{list2}{
  \begin{list}{$\bullet$}{%
      \setlength{\itemsep}{0in}
      \setlength{\parsep}{0in} \setlength{\parskip}{0in}
      \setlength{\topsep}{0in} \setlength{\partopsep}{0in}
      \setlength{\leftmargin}{0.2in}}}{\end{list}}


\begin{document}
\begin{flushleft}
		\huge \textbf{Kristin M. Eccles}
		\hfill {\LARGE \textbf{Curriculum Vitae}}\\
\end{flushleft}


\begin{resume}
\section{\sc Contact Information}
\vspace{.05in}
\begin{tabular}{@{}p{3.2in}p{2.5in}}
Exposure and Biomonitoring Division & {\it Voice:}  (613) 816-1041 \\
Health Canada & {\it E-mail:}  kristin.eccles@hc-sc.gc.ca\\
251 Sir Frederick Banting Driveway&{\it Website:} https://kristineccles.wordpress.com  \\
Sir Frederick Banting Building&{\it GitHub:} https://github.com/kristineccles \\
Ottawa, ON, K1A 0K9 & {\it ORCID}: 0000-0002-6629-430X

\end{tabular}

\vspace*{.1in}
\section{\sc Highlights}
23 peer-reviewed publications (10 as first author)
\\
327 citations; h index = 10 and i10 index = 10
\\
Research interests: Exposome, biomarkers, NAMs, geospatial modeling

\vspace*{.1in}
\section{\sc Current\\Position}
{\bf  Research Scientist} \hfill {\bf July 2023 - Present}\\
Principle Investigator, Computational Toxicology Group\\
Exposure and Biomonitoring Division, Health Canada\\
Ottawa, Ontario, Canada

%%%%%%% Education %%%%%%%
\vspace*{.1in}
\section{\sc Education}

\textbf{Ph.D., Biology with Specialization in Chemical and Environmental Toxicology} \hfill {\bf 2019}\\
Department of Biology, University of Ottawa, Ottawa, Canada \\
Advisor: Laurie Chan, Ph.D.

\textbf{M.Sc., Geography}  \hfill {\bf 2014}\\
Department of Geography, University of Calgary, Calgary, Canada\\
Advisors:  Stefania Bertazzon, Ph.D. and Sylvia Checkley, Ph.D.

\textbf{Honours B.A., Major: Health Studies, Minors: Geography and Earth Science }\hfill {\bf 2012}\\
McMaster University, Hamilton, Canada\\
Advisor: John Eyles, Ph.D.

%%%%%%% Appointments %%%%%%%
\vspace*{.1in}
\section{\sc Professional Appointment/ Employment}

{\bf  Postdoctoral Research Fellow} \hfill {\bf Nov 2020 - May 2023}\\
National Institute of Environmental Health Science, Division of Translational Toxicology,\\
Durham, North Carolina, USA\\
Postdoctoral Advisors: Cynthia Rider, Ph.D and Kyle Messier, Ph.D.

{\bf Postdoctoral Fellowship} \hfill {\bf Aug 2019 - Oct 2020}\\
Department of Geography, Geomatics and Environment, University of Toronto, Mississauga, Canada\\
Postdoctoral Advisors: Igor Lehnherr, Ph.D. and Trevor Porter, Ph.D.

{\bf Geomatics Researcher} \hfill {\bf June 2017 - March 2019}\\
National Wildlife Researcher Center, Environment and Climate Change Canada, Ottawa, Canada
\vspace*{.1in}

%%%%%%% Publications %%%%%%%
\section{\sc Peer-reviewed Publications}

Vander Meulen, I. J., Schock, D. M., Akhter, F., Mundy, L. J., \textbf{Eccles K.M.}, Soos, C., Peru, K.M., McMartin, D.W., Headley, J.V. and Pauli, B.D. (2023). Site-specific spatiotemporal occurrence and molecular congener distributions of naphthenic acids in Athabasca oil sands wetlands of Alberta, Canada. \textit{Environmental Pollution}, 122061. https://doi.org/10.1016/j.envpol.2023.122061 \\
\\

Tommasi, F., Pagano, G., Oral, R., Thomas, P.J., \textbf{Eccles K.M.}, Tez, S., Toscanesi, M., Giarra,, A., Siciliano, A., Dipierro, N., Gjata, I,, Guida, M., Libralato, G., Lyons,D.M., Burić, P., Ines Kovačić, I.,  Trifuoggi, M. Topsoil pollution and multi-endpoint toxicity in the petrochemical area of Augusta-Priolo (eastern Sicily, Italy). \textit{Chemosphere}, 333, 138802. https://doi.org/10.1016/j.chemosphere.2023.138802\\

\textbf{Eccles K.M.}, Karmaus, A. L., Kleinstreuer, N. C., Parham, F., Rider, C. V., Wambaugh, J. F., Messier, K. P. (2023). A geospatial modeling approach to quantifying the risk of exposure to environmental chemical mixtures via a common molecular target. \textit{Science of The Total Environment}, 855, 158905. https://doi.org/10.1016/j.scitotenv.2022.158905\\
\mbox{*}NIEHS 2022 paper of the year

Boutet, V., Dominique, M., \textbf{Eccles, K.M.}, Branigan, M., Dyck, M., van Coeverden de Groot, P., Lougheed, S.C., Rutter A., Langlois V.S. An exploratory spatial contaminant assessment for polar bear (\textit{Ursus maritimus}) liver, fat, and muscle from Northern Canada. (2023). \textit{Environmental Pollution},  316, 120663. https://doi.org/10.1016/j.envpol.2022.120663

Lowe, M.E., Akhtari, F., Potter, P.A., Fargo, D.C., Schmitt, C.P., Schurman, S.H., \textbf{Eccles, K.M.}, Motsinger-Reif, A., Hall, J.E., Messier, K.P. (2022). The skin is no barrier to mixtures: Air pollutant mixtures and reported
psoriasis or eczema in the Personalized Environment and Genes Study (PEGS). \textit{Journal of exposure science \& environmental epidemiology}.1-8. https://doi.org/10.1038/s41370-022-00502-0

Cui, Y., \textbf{Eccles K.M.}, Kwok, R.K., Joubert, B., Messier, K.P., Balshaw, D. (2022). Integrating Multiscale Geospatial Environmental Data into Large Population Health Studies: Challenges and Opportunities. \textit{Toxics}. 10(403). https://doi.org/10.3390/toxics1007040

Thomas, P. J., Eickmeyer, D. C., \textbf{Eccles, K.M.}, Kimpe, L. E., Felzel, E., Brouwer, A., Blais, J. M. (2022). Paleotoxicity of petrogenic and pyrogenic hydrocarbon mixtures in sediment cores from the Athabasca oil sands region, Alberta (Canada). \textit{Environmental Pollution}, 292, 118271. \\
https://doi.org/10.1016/j.envpol.2021.118271

\textbf{Eccles, K.M.}, Thomas, P. J., Chan, H. M. (2021). Spatial patterns of the exposure-response relationship between mercury and cortisol in the fur of river otter (\textit{Lontra canadensis}). \textit{Chemosphere}, 263, 127992. https://doi.org/10.1016/j.chemosphere.2020.127992

Thomas, P. J., Newell, E. E., \textbf{Eccles, K.M.}, Holloway, A. C., Idowu, I., Xia, Z., Quenneville, C. (2021). Co-exposures to trace elements and polycyclic aromatic compounds (PACs) impacts North American river otter (\textit{Lontra canadensis}) baculum. \textit{Chemosphere}, 265, 128920.\\https://doi.org/10.1016/j.chemosphere.2020.128920

\textbf{Eccles, K.M.}, Pauli, B.D., Chan, H.M. (2020). Geospatial analysis of complex metal exposures to biota in the Athabasca Oil Sands. \textit{PLoS one}, 15(9), e0239086. \\https://doi.org/10.1371/journal.pone.0239086

Galen, G., \textbf{Eccles, K.M}., MacMillian, M., Thomas, P. J., Chan, H.M., Poulain, A.J. (2020). The gut microbial community structure of the North American river otter (\textit{Lontra canadensis}) in the Alberta Oil Sands Region in Canada: relationship with local environmental variables and metal body burden.\textit{Environmental Toxicology and Chemistry}.39(12), 2516-2526. https://doi.org/10.1002/etc.4876

Etowa, J., Johnston, A., Jama, Z., \textbf{Eccles, K.M.}, Ashton, A. (2020). Mixed-method evaluation of a community-based postpartum support program: a study protocol. \textit{BMJ open}, 10(10), e036749. https://doi.org/10.1136/bmjopen-2019-036749

\textbf{Eccles, K.M.}, Majeed, H., Lehnherr, I., Porter, T. (2020). A continental and marine-influenced tree-ring mercury record in the Old Crow Flats, Yukon, Canada. \textit{ACS Earth and Space Chemistry}, 4(8), 1281-1290. https://doi.org/10.1021/acsearthspacechem.0c00081.s001

Cheney, C.L., \textbf{Eccles, K.M}., Kimpe, L.E., Blais, J.M. (2020). Determining the effects of past gold mining using a sediment palaeotoxicity model. \textit{Science of The Total Environment}, 718, 137308. https://doi.org/10.1016/j.scitotenv.2020.137308

\textbf{Eccles, K.M.}, Thomas, P. J., Chan, H. M. (2020). Relationships between mercury concentrations in fur and stomach contents of river otter (\textit{Lontra canadensis}) and mink (\textit{Neovison vison}) in northern Alberta Canada and their applications as proxies for environmental factors determining mercury bioavailability. \textit{Environmental Research}, 181, 108961. https://doi.org/10.1016/j.envres.2019.108961

\textbf{Eccles, K. M}., Pauli, B. D., Chan, H. M. (2019). The use of Geographic Information Systems (GIS) for spatial ecological risk assessments: An example from the Athabasca oil sands area in Canada. \textit{Environmental toxicology and chemistry}, 38(12): 2797–2810. https://doi.org/10.1002/etc.4577

\textbf{Eccles, K. M.}, Littlewood, E. S., Thomas, P. J., Chan, H. M. (2019). Distribution of organic and inorganic mercury across the pelts of Canadian river otter (\textit{Lontra canadensis}). \textit{Scientific reports}, 9(1), 3237. https://doi.org/10.1038/s41598-019-39893-w

\textbf{Eccles, K. M.}, Thomas, P. J., Chan, H. M. (2017). Predictive meta-regressions relating mercurytissue concentrations of freshwater piscivorous mammals. \textit{Environmental Toxicology and Chemistry}, 36(6), 2377–2384. http://doi.org/10.1002/etc.3775

Thomas, P. J., \textbf{Eccles, K. M.}, Mundy, L. J. (2017). Spatial modelling of non-target exposure to anticoagulant rodenticides can inform mitigation options in two boreal predators inhabiting areas with intensive oil and gas development. \textit{Biological Conservation}, 212, 111-119. \\https://doi.org/10.1002/etc.3775

Hu, X. F., \textbf{Eccles, K. M.}, Chan, H. M. (2017). High selenium exposure lowers the odds ratios for hypertension, stroke, and myocardial infarction associated with mercury exposure among Inuit in Canada. \textit{Environment International}, 102, 200-206. https://doi.org/10.1016/j.envint.2017.03.002

\textbf{Eccles, K. M.}, Checkley, S., Sjogren, D., Barkema, H. W., Bertazzon, S. (2017). Lessons learned from the 2013 Calgary flood: Assessing risk of drinking water well contamination. \textit{Applied Geography}, 80, 78-85. https://doi.org/10.1016/j.apgeog.2017.02.005

\textbf{Eccles, K.M.}, Bertazzon, S. (2015). Applications of geographic information systems in public health: A geospatial approach to analyzing MMR immunization uptake in Alberta. \textit{Canadian Journal of Public Health}, 106(6). https://doi.org/10.17269/cjph.106.4981

Bertazzon, S., Johnson, M., \textbf{Eccles, K.}, Kaplan, G. G. (2015). Accounting for spatial effects in land use regression for urban air pollution modelling. \textit{Spatial and Spatio-temporal Epidemiology}. 14-15, 9–21. https://doi.org/10.1016/j.sste.2015.06.002

\vspace*{.1in}
\section{\sc Conference Proceedings}
\textbf{Eccles K.M.}, Thomas P.J., Chan H.M. (2016). Evaluating mercury guidelines for furbearers using a predictive meta-model. Canadian Ecotoxicity Workshop. Edmonton, Canada.

Bertazzon, S., Barrett, O., Johnson, \textbf{M., Eccles,} K, Zhang, J. Y. (2014). Land use regression models (LUR) for reliable estimation of air quality in Calgary. Spatial Knowledge and Information. Banff, Canada.\\

\vspace*{.1in}
\section{\sc Invited Talks}

\textbf{Eccles K.M.}(2023). From Molecules to Maps: Assessing spatial patterns of contaminant sources, exposures, and health effects on humans and wildlife. Health Canada. Ottawa, Canada.

\textbf{Eccles K.M.}(2022). From Molecules to Maps: Assessing spatial patterns of contaminant sources, exposures, and health effects on humans and wildlife. Rutgers University. Newark, New Jersey.

\textbf{Eccles K.M.}(2020). From biomarkers to biomes: Relationships between contaminant sources, exposures, and health outcomes. University of Toronto Intersectional Seminar Series. Toronto, Ontario.

\textbf{Eccles K.M.}(2020). Humans, wildlife, and the environment: Assessing ecological health. 2nd Annual GeoHealth Network Conference. Toronto, Ontario. (Cancelled due to COVID-19)

\textbf{Eccles K.M.},Chan H.M. (2018). Mercury in wild foods and food security: Integrating data (Presentation). Environment and Climate Change Canada (ECCC) Wildlife Division Health Division Annual Meeting. Ottawa, Ontario.

\textbf{Eccles K.M.}, Chan H.M. (2018). Modelling the relationship between contaminant sources and exposures in wildlife (Presentation). Environment and Climate Change Canada (ECCC) National Pollution Release Inventory (NPRI) Data Users Workshop. Ottawa, Ontario.

%%%%%%% Conferences %%%%%%%
\vspace*{.1in}
\section{\sc Selected Conference Presentations (12/24)}

\textbf{Eccles K.M.}, Karmaus, A. L., Kleinstreuer, N. C., Parham, F., Rider, C. V., Messier, K. P. (2023). Mapping a Path to Disease: Quantifying the risk of exposure to environmental chemical mixtures via a common molecular target using a geospatial modeling approach (Presentation). Society of Toxicology, Nashville, USA.\\
$^{*}$1st place winner of best postdoctoral abstract for the SOT Mixtures specialty section

\textbf{Eccles K.M.}, Rider, C. V., Messier, K. P. (2022). Geospatial Risk Assessment Using High-Throughput Screening Assays To Quantify Potential Adverse Effects From Exposure To Chemical Mixtures (Presentation). Society of Environmental Toxicology and Chemistry, Pittsburgh, USA.

\textbf{Eccles K.M.}, Karmaus, A. L., Kleinstreuer, N. C., Parham, F., Rider, C. V., Wambaugh, J. F., Messier, K. P. (2022). A geospatial modeling approach to quantifying the risk of exposure to environmental chemical mixtures via a common molecular target (Poster). North Carolina Society of Toxicology, Durham, USA.$^{*}$\\
$^{*}$1st place winner of best postdoctoral poster and presentation

\textbf{Eccles K.M.}, Messier, K.P, (2021). Geospatial Risk Characterization Mapping of Chemical Mixtures Through Connections to Toxicological Adverse Outcome Pathways (Presentation). American Geophysical Union, New Orlean, USA.

\textbf{Eccles K.M.}, Kleinstreuer, N.C., Wambaugh, J.F., Messier, K.P, (2021). A geospatial modeling approach to quantifying risk of exposure to environmental chemical mixtures via a common molecular initiating event (Poster). International Society of Environmental Epidemiology, New York, USA.

\textbf{Eccles K.M.}, Clackett A., Ghotra, A., Majeed, I., Lehnherr, I., Porter, T. (2020). Developing a network of historical atmospheric mercury trends using tree-rings in northern Canada (Presentation).  Society of Environmental Toxicology and Chemistry, Fort Worth, USA.

\textbf{Eccles K.M.}, Clackett A., Ghotra, A., Majeed, I., Lehnherr, I., Porter, T. (2019). Assessing variability of atmospheric mercury (Hg$^{0}$) trends using tree-rings in northern Canada (Presentation). Society of Environmental Toxicology and Chemistry. Toronto, Canada.

\textbf{Eccles K.M.}, Thomas P.J., Chan H.M. (2019). Wildlife as a surrogate indicator for impacts of mercury on ecosystem health (Presentation). International Conference on Mercury as a Global Pollutant. Krakow, Poland.

\textbf{Eccles K.M.}, Thomas P.J., Chan H.M. (2018). Wildlife as a surrogate indicator for impacts of mercury on ecosystem health (Presentation). Society of Environmental Toxicology and Chemistry. Sacramento, USA.

\textbf{Eccles K.M.}, Thomas P.J., Chan H.M. (2018). Evaluating the co-dispersion of mercury sources and wildlife exposures in the Athabasca Oil Sands region (Presentation). Society of Environmental Toxicology and Chemistry. Sacramento, USA.

\textbf{Eccles, K.M,} Hebert C.E., Schock, D., Akhter F., Mundy L., Thomas P.J., Pauli, B.D. (2018). Evaluating the co-dispersion of mercury sources and wildlife exposures in the Athabasca Oil Sands region (Presentation). Society of Environmental Toxicology and Chemistry. Sacramento, USA.

\textbf{Eccles K.M.}, Thomas P.J., Chan H.M. (2018). Using geospatial methods to quantify the co-dispersion of mercury sources and exposures in river otter (\textit{Lontra canadensis}) for risk prediction (Presentation). International Society of Exposure Science and International Society of Environmental Epidemiology Joint Meeting. Ottawa, Canada.

%%%%%%% Teaching %%%%%%%

\vspace*{.1in}
\section{\sc Teaching Experience}
\textbf{Primary Instructor}\\
Graduate Level Short Course: Introduction to R in Open-Source Methods \hfill {\textbf{Fall 2020}}\\
Department of Geography, Geomatics and Enviornment, University of Toronto \hfill {\textbf{Winter 2020}}

Geographic Information Systems \hfill {\textbf{Spring 2020}}\\
Department of Geography, Geomatics and Enviornment, University of Toronto

Introduction to Quantitative Methods
 \hfill {\textbf{Winter 2018}}\\
Department of Geography and Environmental Studies, Carleton University

Mapping and Modelling the Real World: Introduction to GIS \hfill {\textbf{May 2017}}\\
Enrichment Mini-Course, University of Ottawa

Introduction to Geomatics
\hfill {\textbf{Fall 2016}}\\
Department of Geography, Environment and Geomatics, University of Ottawa

\textbf{Teaching Assistant }\\
University of Ottawa, Ottawa, ON
\hfill {\textbf{2014 - 2017}}\\
Spatial Ecology, Biostatistics, Environmental Science

\vspace*{.1in}

%%%%%%% Awards %%%%%%%
\section{\sc Competitive Awards}

Society of Toxicology (SOT) Mixtures Specialty Section\\
Best Postdoctoral Abstract (2023)
\hfill {\textbf{Recognition}}\\
Society of Toxicology (SOT) Biological Modeling Specialty Section\\
Andersen-Clewell Trainee Award - 2nd Place (2023)
\hfill {\textbf{Recognition}}\\
NIEHS Paper of the Year (2022)
\hfill {\textbf{Recognition}}\\
North Carolina Society of Toxicology (NCSOT)\\
Best Postdoctoral Poster and Presentation (2022)
\hfill {\textbf{\$300}}\\
SETAC Travel Award (2022)
\hfill {\textbf{\$1050}}\\
University of Toronto Postdoctoral Award (2019-2020)
\hfill {\textbf{\$45,000}}\\
NSERC CREATE-REACT (2016 - 2018)
\hfill {\textbf{\$20,000}}\\
NSERC CREATE-REACT Travel Award (2018)
\hfill {\textbf{\$5,000}}\\
University of Ottawa Excellence Scholarship (2016 - 2017)
\hfill {\textbf{\$8,200}}\\
Queen Elizabeth II Graduate Scholarship in Science and Technology (2016 - 2017)
\hfill {\textbf{\$15,000}}\\
University of Ottawa Entrance Scholarship (2014 - 2018)
\hfill {\textbf{\$38,000}}\\

\vspace*{.1in}

\section{\sc Leadership and Service}
{\bf Conference Sessions and Workshops Delivered}\\
Society of Environmental Toxicology and Chemistry, Fort Worth, USA \hfill {\bf  Nov 2020}\\
On Demand Session: Mercury emissions, transport, and transformation in a changing environment\\
Live Discussion: Pathways between Hg sources and exposures in a changing world\\
Workshop: Introduction to R

International Conference on Mercury as a Global Pollutant, Krakow, Poland \hfill {\bf Sept 2019}\\
Workshop: Latest Advances in Wildlife Biomonitoring

{\bf Expert Working Group Member}\\
Arctic Monitoring Assessment Program (AMAP)
\hfill {\bf  June 2019- Sept 2020}\\
Mercury Expert Working Group

Oil Sands Monitoring Integration Workshop Series
\hfill {\bf  Jan 2019}\\
External Expert for Geospatial Analysis and Mercury

\vspace*{.1in}

\section{\sc Additional Training}
Training in the Responsible Conduct of Research, National Institutes of Health \hfill {\bf  Fall 2021}\\
Teaching Fundamentals Certificate, University of Toronto
\hfill {\bf  Winter 2020}\\
Machine Learning, University of Toronto
\hfill {\bf  Fall 2019}

\vspace*{.1in}

\section{\sc Languages}

English - Native Language, French - Good\\
R - Advanced, Python - Intermediate, LaTeX- Intermediate

\vspace*{.1in}

\section{\sc Professional Memberships}
Society of Toxicology (SOT)\\
Data Visualization Society\\


\end{resume}
\end{document}




